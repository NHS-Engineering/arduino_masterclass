\documentclass{beamer}

\usepackage{xspace}
\usepackage{ulem}
\usepackage{listings}

\title{Arduino Masterclass}
\subtitle{(also general programming and stuff)}
\date{January 19, 2023}

\newcommand{\cee}{\texttt{C}\xspace}
\newcommand{\cxx}{\texttt{C++}\xspace}
\newcommand{\examplefile}[1]{
	\lstinputlisting[language=C++, basicstyle=\tiny]{examples/#1}
}

\usetheme{Darmstadt}
\begin{document}
	\begin{frame}
		\titlepage
	\end{frame}

	\begin{frame}
		\tableofcontents
	\end{frame}

	\section{Arduino 101}
	\subsection{The Basics}
	\begin{frame}{What Is an Arduino?}
		An Arduino is essentially a really small and power efficient computer with a few tradeoffs.

		\begin{itemize}
			\item an Arduino is a microcontroller (MCU $\neq$ CPU)
			\begin{itemize}
				\item severely restricted RAM, ROM, and EEPROM
				\item no operating system
				\item limited graphical capabilities
			\end{itemize}
			\item interfaces with electronics via pins
		\end{itemize}
	\end{frame}

	\begin{frame}{What Can I Do With an Arduino?}
		\begin{itemize}
			\item control electronics
			\begin{itemize}
				\item respond to stimuli from a sensor
				\begin{itemize}
					\item light
					\item sound
					\item temperature
					\item GPS location
				\end{itemize}

				\item control other electronics
				\begin{itemize}
					\item motor
					\item servo
					\item solonoid
					\item \sout{lasers} LEDs
				\end{itemize}
			\end{itemize}
		\end{itemize}
	\end{frame}

	\section{The Arduino Language}
	\subsection{Programming Introduction}
	\begin{frame}{How Do I Program an Arduino?}
		Arduinos are (typically) programmed using the Arduino IDE in the Arduino Langauge
		\begin{itemize}
			\item the Arduino Language is essentially \cxx without the boilerplate
			\item \cxx is a superset of \cee
			\item most Arduino programming is essentially \cee
			\item most embedded systems use \cee, so most of the knowledge here will transfer to the real world
			\item \sout{Rust is cooler}
		\end{itemize}
	\end{frame}

	\begin{frame}{Note on Compilers}
		\begin{itemize}
			\item programs are text which are passed to a compiler which translates it into machine code
			\item what you see $\neq$ what you get
			\item you will not saves bytes by removing blank lines
			\item sometimes your code can be automatically optimized for speed and/or size
			\item it will complain if you don't end statements with a semicolon
		\end{itemize}
	\end{frame}

	\subsection{Basic Structure}
	\begin{frame}[fragile]{Structure of a Program}
		\examplefile{structure.cxx}
	\end{frame}

	\subsection{Variables}
	\begin{frame}[fragile]{Variables}
		\begin{itemize}
			\item variables denote a location in RAM in which data of a specified type can be stored
			\item it's all just binary, but the type gives the compiler information on how that data is interpreted
			\item declared with \verb|<type> <name> = <value>;|
			\item assigned (after declaration) with \verb|<name> = <value>;|
			\item accessed with their name
		\end{itemize}
	\end{frame}

	\begin{frame}[fragile]{Data Types}
		pretty much the same as \cxx if you're familiar with that
		\begin{itemize}
			\item \verb|bool|: a boolean value, either \verb|true| or \verb|false|
			\item \verb|byte|: an unsigned 8 bit integer ($0$ to $255$)
			\item \verb|int|: an signed integer of platform defined size (in our case 16 bits, so between approximately $\pm30,000$-ish)
			\item \verb|long|: a beeg boy signed 32 bit integer (between about $\pm2$ billion)
			\item \verb|float|: a decimal number (ex: $1, 1.0, 1.2, -3.14$)
			\item \verb|double|: a bigger \verb|float| of platform defined size, sometimes identical to \verb|float|
		\end{itemize}
	\end{frame}

	\begin{frame}{Operators}
		TODO: write
	\end{frame}

	\begin{frame}[fragile]{Example}
		\examplefile{variables.cxx}
	\end{frame}

	\subsection{Functions}
	\begin{frame}{Functions}
		TODO: write
	\end{frame}

	\subsection{Control Flow}
	\begin{frame}{Control Flow}
		TODO: write
	\end{frame}

	\subsection{Arrays}
	\begin{frame}{Control Flow}
		TODO: write
	\end{frame}

	\section{Conclusion}
	\subsection{Questions?}
	\begin{frame}{Thank You for Listening}
		any questions?
	\end{frame}

	\subsection{Further Reading}
	\begin{frame}{Additional Cool Stuff}
		TODO: include links to Rust stuff
	\end{frame}
\end{document}
